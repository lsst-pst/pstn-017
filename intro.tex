\section{Introduction}

Within Rubin Observatory DM was tasked to stand up operable, maintainable, quality services to deliver high-quality LSST data products for science and education, all on time and within reasonable cost.
DM is responsible for provided the tools necessary to take the bits generated by the telescope and turn them in to science ready products.

 See also The Ruben Observatory  Data Management System: \cite{2015arXiv151207914J}


\subsection{Science Drivers}
The astronomical size and complexity of the expected Rubin data drives many of the architectural choices made for the DM system. Teh following table highlights some of the key numbers that have influenced choices in DM.

\begin{table}[ht]
\caption{Rubin Key Numbers driving DM architectural choices}
\label{tab:dm_keynumbers}
\begin{center}
\begin{tabular}{|l|l|c|} %% this creates two columns

\hline
\rule[-1ex]{0pt}{3.5ex}  {\bf Parameters} & {\bf Number} & {\bf Unit}\\
\hline
N Objects & 40 billion & -- \\
N Alerts per image & 10 000 & -- \\
N Alerts per night & 10 million & -- \\
\hline
\end{tabular}
\end{center}
\end{table}

