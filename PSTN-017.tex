\documentclass[modern]{aastex62}

% lsstdoc documentation: https://lsst-texmf.lsst.io/lsstdoc.html
% GENERATED FILE -- edit this in the Makefile
\newcommand{\lsstDocType}{PSTN}
\newcommand{\lsstDocNum}{017}
\newcommand{\vcsRevision}{7a71ec3-dirty}
\newcommand{\vcsDate}{2020-01-22}


% Package imports go here.

% Local commands go here.



\newcommand{\docRef}{PSTN-017}
\newcommand{\docUpstreamLocation}{\url{https://github.com/lsst-pst/pstn-017}}

\providecommand{\secref}[1]{Section~\ref{#1}}
\providecommand{\appref}[1]{Appendix~\ref{#1}}
\providecommand{\partref}[1]{Part~\ref{#1}}
\providecommand{\tabref}[1]{Table~\ref{#1}}
\providecommand{\figref}[1]{Figure~\ref{#1}}
\providecommand{\eqnref}[1]{Eq.~\ref{#1}}


\begin{document}
%% DO NOT EDIT THIS FILE. IT IS GENERATED FROM db2authors.py"
%% Regenerate using:
%%    python $LSST_TEXMF_DIR/bin/db2authors.py > authors.tex


\author[0000-0003-4141-6195]{William~O'Mullane}
\affiliation{Rubin Observatory Project Office, 950 N.\ Cherry Ave., Tucson, AZ  85719, USA}

\author[0000-0003-0800-8755]{Leanne~P.~Guy}
\affiliation{Rubin Observatory Project Office, 950 N.\ Cherry Ave., Tucson, AZ  85719, USA}

\author[0000-0001-9445-1846]{John~D.~Swinbank}
\affiliation{University of Washington, Dept.\ of Astronomy, Box 351580, Seattle, WA 98195, USA}
\affiliation{Department of Astrophysical Sciences, Princeton University, Princeton, NJ 08544, USA}

\author[0000-0002-0558-0521]{Colin~T.~Slater}
\affiliation{University of Washington, Dept.\ of Astronomy, Box 351580, Seattle, WA 98195, USA}


\date{\today}
\title{Overview of LSST Data Management}
\hypersetup{pdftitle={\@title}, pdfauthor={\@author}, pdfkeywords={\@keywords}}


\begin{abstract}
Vera C. Rubin Observatory Data Management (DM) subsystem is one of four construction subsystems.
In operations we retain the notion of four subsystems and DM remains as one of them.

overview ...
\end{abstract}




\section{Introduction}

Within the Vera C.\ Rubin Observatory \citep{2019ApJ...873..111I} the Data Management (DM) team was tasked to stand up operable, maintainable, quality services to deliver high-quality LSST data products for science and education, all on time and within reasonable cost.
DM is responsible for provided the tools necessary to take the bits generated by the telescope and turn them in to science ready products.

 See also the Rubin Observatory  Data Management System \citep{2017ASPC..512..279J,2022arXiv221113611O}


\subsection{Science Drivers}
The astronomical size and complexity of the expected Rubin data drives many of the architectural choices made for the DM system. The following table highlights some of the key numbers that have influenced choices in DM.

\begin{deluxetable}{llc}
\tablecaption{Rubin Key Numbers driving DM architectural choices\label{tab:dm_keynumbers}}

\tablehead{\colhead{Parameters} & \colhead{Number} & \colhead{Unit}}

\startdata
N Objects & 40 billion & -- \\
N Alerts per image & 10 000 & -- \\
N Alerts per night & 10 million & -- \\
N Images per night & 1000 & -- \\
\enddata

\end{deluxetable}

\subsection{Technical Challenges}
The operational goal of Rubin Observatories Legacy Survey of Space and time is to produce an optical/near-IR survey of half the sky in ugrizy bands to r 27.5 (36 nJy) based on 825 visits over a 10-year period. It is a deep wide fast survey.
Each Rubin image is around 8GB and we take more than one per minute or about 1000 per night.
Add the $\approx$ 450 calibration exposures each day and it means about 20TB of data has to be shipped from Chile to SLAC on a daily basis.
Alerts are to be produced in under 2 minutes with a  goal of doing them in 1 minute which gives us a
challenging transmission and prompt processing time window (see \autoref{sec:dataacquisition}).

Over the 10 year survey we estimate having 2.75M on sky images and at least 1M more calibrations totaling about 50PB.
As we grow this must be reprocessed each year to produce the data releases (see \autoref{sec:dataproduction}.



\section{Organisation of Data Management} \label{sec:org}
Org chart meetings etc. from \cite{LDM-294}.

\subsection{Relationship to other subsystems}
   We take images from the  LSST camera: \cite{2010SPIE.7735E..0JK}

   We are commanded and listen to the  Telescope  and site software  \cite{2014SPIE.9145E..1AG}


   DM will be verified and validated as part of System verification and validation: \cite{2014SPIE.9150E..0NS}

\section {Architecture, Data Transmission and  Access } \label{sec:dataacess}
System vision diagram
Show network diagram and architecture diagrams for alerts/DRP. Mention Data Access doc  \cite{LDO-013} and
setting up of DACs in Chile and USA.

\section {Software Products} \label{sec:softproducts}
The DM products are not data as many may think, rather the products are software and services to produce those
products.
In operations DM will not exist though an analogous organization called Data Production will spring into existence under similar leadership.


The high level list of DM products is given in \figref{fig:pt}, as may be seen in the  figure we consider software, services and systems (enclaves) as products.


\begin{figure}
\begin{centering}
\includegraphics[width=0.9\textwidth]{images/ProductTree}
	\caption{Rubin DM product tree \label{fig:pt}}
\end{centering}
\end{figure}



\subsection{Commissioning Software Products}
A detailed section describing what we delivered and used to process the commissioning data.
This section will probably not be written until  commissioning.

\subsection{Anticipated Data Release 1 Software Products}
A short section describing what we anticipate delivering  for DR1 in addition to what was delivered in commissioning.
This section will probably not be written until after commissioning.

\subsection{Science Pipelines}\label{sec:pipes}
The science pipelines which produce the prompt and data release products are of course the most identifiable
software product from DM.
The detailed approach to Science Pipelines is covered in \cite{PSTN-019}.


\subsubsection{Science Platform}

\subsubsection{Services}

\section{Data Products} \label{sec:dataproducts}

The LSST data collected by Rubin Observatory is automatically processed by the LSST science pipelines, as described in (\S~\ref{sec:softproduts})  to produce the LSST data products. The  reduced images, catalogs and alerts. 

\subsection{Types of Data Product} \label{sec:dp-categories}
LSST produces three types of data products. 
\begin{itemize}
\item images:  raw single visit images, calibrated processed visit images (PVI), coadd images, cutouts (postage stamps) 
\item catalogs: DR includes Object, Source, DIASource, DIAObject, 
\item alerts: A composite data product that incliudes:  An alert packet includes:  . \end{itemize}


Alerts packets are distributed via the alert distribution system (\S~ref) for  all oobjects  have changed in brightness or position on the sky.

Describe the sceince that they will enable - i.e why are we creating these data products? How are they created, how are they distributed and or served


Generation of intermediate data products, in particular to generate intermediate flavours of coadds. 

 The detailed  Data Products Definition may be found in  \cite{LSE-163}.


\subsection{Data Product Categories} \label{sec:dp-categories}
LSST defines three main categories of data products to be served by Rubin. 
Each category may comprise any or all of the data product types described in \S~

These data product categories are defined in the SRD (\citep{LPM-17}) and have been a driver for DM 
(add more detail about why) 

\begin{itemize}
\item {\t Prompt:} data products are generated continuously every observing night, 
including both alerts to objects that have changed brightness or position, 
which are released with 60-second latency, 
and other catalog and image data products that are released with 24-hour latency. 
Prompt  data products comprise PVIs,  DIASource, DIAObject catalogs, 
\item {\tt Data Release:} data products will be made available as part of an LSST Data Release (\S~\ref{}) as the result of coherent
processings of the entire science data set to date. 
These will include calibrated images;measurements of positions, fluxes, and shapes; variability information such as orbital
parameters for moving objects; and an appropriate compact description of light curves.
The Data Release data products will include a uniform reprocessing of the difference
imaging-based Prompt data products.
\item {\tt User Generated:} data products will originate from the community, including project teams. 
These will be created and stored using suitable Application Programming Interfaces (APIs) 
that will be provided by the LSST Data Management System. 
The system will allow the science teams to use the full power of the Rubin database systems and
Science Platform for the storage, access, and analysis of their results. 
It will provide for users and groups to maintain access control over the data products they create, 
enabling them to have limited distribution or to be shared with the entire LSST community. 
\end{itemize}

% SRD

%data products are generated continuously every observing night, including alerts
%to objects that have changed brightness or position.
%
%data products will be made available as annual Data Releases and will include
%images and measurements of positions, fluxes, and shapes, as well as variability information such as orbital parameters for moving objects and an appropriate compact
%description of light curves.

%data products will be created by the community, including project teams, using
%suitable Applications Programming Interfaces (APIs) that will be provided by the LSST
%Data Management System. The Data Management System will also provide at least 10%
%of its total capability for user-dedicated processing and user-dedicated storage. The key
%aspect of these capabilities is that they will reside ?next to" the LSST data, avoiding the
%latency associated with downloads. They will also allow the science teams to use the
%database infrastructure to store their results.


The first two, {\tt Prompt} and {\tt Data Release} data products are produced and delivered by the DM system described in this paper. 
The third, {\tt User Generated} data products are produced by the Rubin Science Community using the {\tt Prompt} and {\tt Data Release} together possibly with data from other surveys. 

The data product categories are outlined in \cite{LPM-231}

In operations Data Production will use the the software outlined in \secref{sec:softproducts} to produce the various data products.

Show mapping from data product type to category. i.e prompt contains images, catalogs, but not he same ones as DR/ 

UG catalogs can be federated with DR/PP catalogs. 


\subsection{Data Product Verification}
The quality of all Level 1 and Level 2 data products will be extensively assessed, both automatically as well as manually. 
Described in detail in \cite{pstn-024}
Not sure this section is needed. 

DP verification and QA will be carried out automatically following reprocessing runs and in advance of each data release 
Validation of the products is covered in \cite{PSTN-024}.


\subsection{Commissioning Data Products}  \label{sec:dp-commissioning}
Describe here (possibly in a table) the exact data products delivered in commissioning and the science that they enable. 
This will be a high level description and can refer to the DP2 DPDD for exact details. 

\subsection{Anticipated Data Release 1 Data Products} \label{sec:dp-dr1}

The data products delivered during commissioning represent only an initial set of data products. 
The commissioning data set does not enable the production of .... (galaxy shape measurements ... )
In addition to the commissioning data products described in \S~\ref{sec:dp-commissioning}, we anticipate providing: 

\begin{itemize}
\item {\tt Object} catalog photo-$z$ data products.
%\item proper motions for Gaia stars 
\end{itemize}
This list is neither definitive nor exhaustive; the exact list of data products to be provided as part of future LSST Data Releases will be determined closer to the time of release. 

\section {Challenges }
Remaining challenges perhaps ?


\input{conc}



\appendix
% Remove this when you strart your paper
%\input{appendix}
% Include all the relevant bib files.
% https://lsst-texmf.lsst.io/lsstdoc.html#bibliographies
\section{References} \label{sec:bib}
\bibliographystyle{yahapj}
\bibliography{local,lsst,lsst-dm,refs_ads,refs,books}

% Make sure lsst-texmf/bin/generateAcronyms.py is in your path
\section{Acronyms} \label{sec:acronyms}
\addtocounter{table}{-1}
\begin{longtable}{p{0.145\textwidth}p{0.8\textwidth}}\hline
\textbf{Acronym} & \textbf{Description}  \\\hline

DM & Data Management \\\hline
DRP & Data Release Production \\\hline
LDM & LSST Data Management (Document Handle) \\\hline
LDO & LSST Document Operations (Document Handle) \\\hline
LSE & LSST Systems Engineering (Document Handle) \\\hline
LSST & Legacy Survey of Space and Time (formerly Large Synoptic Survey Telescope) \\\hline
PSTN & Project Science Technical Note \\\hline
\end{longtable}


\end{document}
