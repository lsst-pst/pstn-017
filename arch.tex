\section {Architecture, Data Transmission and  Access } \label{sec:arch}
DM spans multiple locations with processing occurring at the USDF (SLAC), FrDF (IN2P3) and UK (ROE).
The system vision has been fairly consistently to deliver science ready data products to the Rubin community as depicted in \autoref{fig:vision}.

\begin{figure*}[ht]
\plotone{DMSystemVision}
\caption{Overview of data management from the telescope to the user. \label{fig:vision}}
\end{figure*}

The organisation and management of DM is covered in \autoref{sec:org}

The DM system architecture was laid out in \cite{LDM-48} from which we reproduce \autoref{fig:arc}
\begin{figure}
\plotone{DMS_Architecture}
\caption{Rubin DM architecture diagram \citet{LDM-148}\label{fig:arch}}
\end{figure}


As shown in \figref{fig:vision} there are several kinds of Rubin data - mostly they are accessed via
the science platform or other services which are described in \autoref{sec:dataservices}.

Data production \autoref{dataproduction} is responsible for all of the pipelines and their execution.

Of course all these services and pipelines must run on hardware which is typically at a data facility.
The data facilities are covered in \autoref{sec:datafacilities}


\subsection{Alerts and Brokers}
Alerts are  product of DM operations and  briefly covered in \autoref{sec:dataproducts}.
The software producing alerts, know and the Alerts Pipeline (AP), is discussed in \todo{REFER TO ALERTS PIPELINES SECTION}
\todo{Leanne: you said you might have a go a this }
Mention community alert brokers.
