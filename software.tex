\section {DM Software Systems} \label{sec:softproducts}
The DM products are not data as many may think, rather the products are software and services to produce those
products.
The management and organisation of DM change slightly for operations (see \autoref{sec:transition}) but many of the same people have similar operations roles
giving a good continuity.
Going into operations we assessed the way DM works and reconceptualized the organisation around the data flow and cyber infrastructure.

The high level list of DM products is given in \figref{fig:pt}, as may be seen in the  figure we consider software, services and infrastructure as our categories of products.

\begin{figure*}[ht]
\plotone{DataMakingServing}
\caption{Rubin DM  organisation in terms of data taking to data serving supported by cyber infrastructure. \label{fig:DataMakingServing}}
\end{figure*}
% https://docs.google.com/presentation/d/1Hcmjoh4ynfV4xbBbVSJ_0C-iAZc1xb5Vb-ceik3uh9g/edit#slide=id.g1c5060272ed_0_0

\begin{figure}
\plotone{ProductTree}
\caption{Rubin DM product tree \label{fig:pt}}
\end{figure}

It must be remarked that these products grew organically to some extent in a less than satisfactory manner.
As mentioned earlier some teams worked within their WBS area and produced planning and products without necessarily paying a lot of attention to other WBS elements. Hence we have some services which are really deployments of software produced by another team e.g Prompt Services and Prompt Software. But we do not always have a service for a piece of software though it may be web accessible and look like a service e.g. QC Products.
Some products are discussed in more below.


\subsection{Data Acquisition} \label{sec:dataacquisition}

\subsection{Data Abstraction} \label{sec:dataabstraction}
text here

\subsection{Data Production} \label{sec:dataproduction}

Data Production is underpinned by the fast and robust LSST Science Pipelines \citet{PSTN-019}, the image processing software written to convert the raw pixels from the Rubin observatory into science-ready data products for astronomers. It takes the raw images as input, and calibrates away the effects of the instrument and atmosphere to produce catalogs and images. Many science analyses can be done with the catalogs alone. Still, as new image processing algorithms are developed over the next decade, we expect the output calibrated coadds, difference images and processed visit images to be used by scientists running specialized detection algorithms during the survey.

The LSST Science Pipelines deliver data products fast and slow. The prompt data products are delivered via the nightly alert stream. These data products support science that requires rapid follow-up. The slower annual data release processing produces calibrated images and catalogs, including lightcurves, to support static sky science and statistical studies of variability.

These pipelines incorporate algorithms for tasks such as detrending, image subtraction, deblending, object characterization, and sky background estimation, among others. When research and development on the pipelines first began in 2004, there was code that could accomplish some of these routines (AstroPy, PyRAF), but none were as robust and fast as needed for LSST. With 3.2 gigapixels of data rolling in every 30 seconds, the data volume grows very quickly. Fast and robust algorithms are needed to process this data efficiently.

The pipelines achieve their speed through Python3-wrapped C++ and are versatile enough for any ground-based optical or IR telescope. However, they require well-sampled PSFs, making them unsuitable for space-based imaging.

The LSST Science Pipelines will continue to evolve throughout LSST's 10-year survey. A portion of LSST's operating budget will be spent on maintaining state-of-the-art algorithms. The state of the art has changed significantly over the last ten years, and there's no reason to believe it will not change over the next decade.

The current algorithms reflect the hard-earned lessons from precursor surveys such as the Dark Energy Survey and Pan-STARRS.
These include, for example:\todo{Yusra check these refs are what you wanted}
\begin{itemize}
\item the PSF modeling algorithm, PIFF (citation)
\item the astrometric calibration algorithms GBDES. \citep{2017PASP..129g4503B}
\item The photometric calibration algorithm, FGCM \citep{2018AJ....155...41B}
\item The artifact rejection algorithm during coaddition. (citation)
\item pattern continuity algorithm for matching amp-to-amp gain offsets (citation?)
\end{itemize}

Formal Agile development practices were adopted in 2014 when we received funding to start construction. At the time, we had minimal-viable algorithm pipelines used in both internal data challenges to process SDSS Stripe 82 data (cite 2012, 2013), and they were also selected as the data release pipelines for the Hyper SuprimeCam Strategic Survey Program  (Bosch et al. 2018, 2019). Feedback from the scientific community, particularly through four public data releases of the Hyper Suprime Cam (HSC) data, has been crucial in refining our algorithms.

We combine unit tests, continuous integration tests, and regression tests. During construction, Jenkins runs continuous integration tests nightly on small subsets of precursor data, including simulated  LSST data and public HSC data. Before merging with the main branch, developers test their ticket branches on these CI tests.

The science pipelines are run in prompt and data release production, utilizing the DM Middleware task framework (CITE DATA ABSTRACTION). This abstraction layer significantly enhances the portability of science pipelines. The Butler acts as a data abstraction layer, removing the need for direct I/O operations or knowledge of the storage backend by the pipelines. Data releases have been successfully executed using the pipelines on Google Cloud and on-premise hardware, managed by workflow systems such as HTCondor or PanDA. The primary startup cost involves ingesting your dataset into the Butler.

All algorithms are implemented as subclasses of the parent PipelineTask, which specifies their inputs and outputs. This structure enables the middleware to construct a directed acyclic graph of all processing tasks required for a specific data product. These tasks are the fundamental building blocks of the pipelines. The pipelines themselves consist of these tasks, each utilized in various ways across different processes. For instance, the data release and other production pipelines include the same subtractImages task.

Initially, the Science Pipelines were designed to run exclusively on CPUs, reflecting the hardware budgeting at the start of construction. Our processes are highly parallelizable, and we anticipate utilizing tens of thousands of cores during data release processing, with each core dedicated to a specific region of the sky or a particular observation. Given the available RAM per core, optimal sizing of sky patches could lead to full CPU utilization. Given advancements in image processing, we are also considering the potential integration of GPUs.

Documentation and installation instructions can be found at pipelines.lsst.io.

\section{Data Services} \label{sec:dataservices}
text here

\subsection{Data Facilities} \label{sec:datafacilities}

As noted in \autoref{sec:arch}, data processing will occur at three data facilities --- in USA, France, and UK. In particular, preparation of the (typically, annual) Data Releases will be distributed across these three facilities using specialised software tools and techniques for distributed data management and remote job submission adopted from the high-energy physics community, with DM providing the required interfaces to the Science Pipeline.

In this arrangement, the USDF will coordinate each processing {\em Campaign} and be the primary curation site, holding a copy of all raw, intermediate, and science-ready products from each production run of the Science Pipeline. The USDF will also be solely responsible for Prompt Processing.

\subsubsection{US Data Facility} \label{sec:ukdf}

\subsubsection{French Data Facility} \label{sec:frdf}

\subsubsection{UK Data Facility} \label{sec:ukdf}

UK interest in the Vera C. Rubin Observatory is coordinated by the LSST:UK Consortium, which has 36 partners representing all major UK astronomy research groups.

Via the Rubin In-kind Contribution program, LSST:UK has proposed --- among other things --- to provide computing resources and associated staff time to undertake $25\%$ of the computing associated with the preparation of each Data Release.

The infrastructure (the UK Data Facility) for this and other significant in-kind contributions has been secured from the UK IRIS programme (www.iris.ac.uk), on a mix of grid, high-performance and research cloud facilities.

In particular, it is proposed that Data Release Processing will occur on grid-computing services at Lancaster University and Rutherford Appleton Laboratories (RAL). Staff at Lancaster and RAL are directly involved in the development of the distributed DRP approach with particular contributions to data distribution and progress tracking, job handling, and infrastructure health monitoring.

LSST:UK has also proposed to operation a full Independent Data Access Center, with capacity to serve the two most recent Data Releases to $20\%$ of the anticipated Rubin international community via the Rubin Science Platform.

The UK IDAC is an integral part of the UK Data Facility, mostly hosted in on-premises cloud resources at the University of Edinburgh, though with some ancillary services provided by RAL. At the time of writing, LSST:UK has been running a prototype IDAC for more than two years, hosting precursor and ancillary astronomy surveys for 20 or so early adopters.

Other contributions that are provided by the UK Data Facility include a Rubin Community Broker, called Lasair and an HPC-based instance of the Science Pipeline for the production of specific User-generated Products that support the fusion of LSST with compatible near-infrared surveys and the crossmatch of LSST object catalogues with contemporary surveys.

